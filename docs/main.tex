\documentclass{article}

\title{The data structure of Clickteam/Multimedia Fusion games}
\author{LAK132}

\begin{document}

\maketitle

This document the the result of many years of hard work by many talented
people, with the goal of exploring the preserving the many fantastic games
built using CTF and MMF.

Special thanks to RED\_EYE (Source Explorer co-developer),
Mathias Kaerlev (Anaconda),
/u/sgzfx (Freddy Explorer),
and to everyone who reported games that broke Source Explorer.

The goal of this document is to outline how CTF and MMF games store their
data, and the methods that Source Explorer uses to read them.

\tableofcontents

% -----------
% Terminology
% -----------

\section{Terminology}

\subsection{Chunk}

Chunks are a discrete block of binary data that hold the data for the game.
Chunks are typically a tightly packed data structures.
Chunks may contain other chunks as their data.

\subsubsection{Identification}

Each chunk has an identification, this typically refers to the type of the
chunk, but may also be the key of the chunk.

\subsubsection{Mode}

Each chunk has a mode, this refers to the compression and encryption methods of
the chunks binary data.

\subsection{Bank}

Bank chunks contain item chunks.
Banks appear to be some kind of unordered map or unordered set.

\subsubsection{Item}

Item chunks are chunks whose identifications do not represent the
type of the chunk, but the key of that chunk in its bank.

\subsection{Frame}

Fusion games are make of a collection of frames.
Frames can be thought of as a scene, a collection of objects and events that
controls the flow of the game.

\subsection{Object}

Sometimes refered to as a `frame item'.

\subsubsection{Instance}

% ----
% Data
% ----

\section{Data}

\subsection{Header codes}
Within the PE header of the game are a few important codes that determine how
the game data should be interpreted.

\subsubsection{Pack}
\verb|HEADER_PACK = 0x1247874977777777|

\subsubsection{PAME}
\label{subsubsec:PAME}
\verb|HEADER_GAME = 0x454D4150 = "PAME"|
\\
If this header code is found, the game is using ASCII strings
(sec~\ref{subsec:string-chunk}).
This is always found in Multimedia Fusion games and sometimes in Clickteam
Fusion games.

\subsubsection{PAMU}
\label{subsubsec:PAMU}
\verb|HEADER_UNIC = 0x554D4150 = "PAMU"|
\\
If this header code is found, the game is using Unicode strings
(sec~\ref{subsec:string-chunk}).
This header is only found in Clickteam Fusion games.

\subsubsection{Product code}

\paragraph{CNC}
\verb|CNCV1VER = 0x0207|

\paragraph{MMF 1.0}
\verb|MMF1 = 0x0300|

\paragraph{MMF 1.5}
\verb|MMF15 = 0x0301|

\paragraph{MMF 2.0}
\verb|MMF2 = 0x0302|

\subsection{Identifications}
See sections \ref{sec:mmf-chunks} and \ref{sec:ctf-chunks} for a list of known
identifications

\subsection{Chunks}
\label{subsec:chunks}

All known chunks take the following form in memory:
\\
\begin{tabular}{c|c|c|c|c|c|c|c}
  0x0 & 0x1 & 0x2 & 0x3 & 0x4 & 0x5 & 0x6 & 0x7
  \\ \hline
  \multicolumn{2}{|c|}{Id} &
  \multicolumn{2}{|c|}{Mode} &
  \multicolumn{4}{|c|}{Chunk size}
  \\ \hline
  \multicolumn{8}{|c}{Data... [Chunk size]}
  \\ \hline
\end{tabular}

\subsubsection{MODE0}
Uncompressed and unencrypted.
\\
\begin{tabular}{c|c|c|c|c|c|c|c}
  0x0 & 0x1 & 0x2 & 0x3 & 0x4 & 0x5 & 0x6 & 0x7
  \\ \hline
  \multicolumn{2}{|c|}{Id} &
  0x00 & 0x00 &
  \multicolumn{4}{|c|}{Chunk size}
  \\ \hline
  \multicolumn{8}{|c}{Data... [Chunk size]}
  \\ \hline
\end{tabular}

\subsubsection[MODE1 (MMF)]{MODE1 (Multimedia Fusion)}
Compressed but unencrypted.
\\
\begin{tabular}{c|c|c|c|c|c|c|c}
  0x0 & 0x1 & 0x2 & 0x3 & 0x4 & 0x5 & 0x6 & 0x7
  \\ \hline
  \multicolumn{2}{|c|}{Id} &
  0x01 & 0x00 &
  \multicolumn{4}{|c|}{Chunk size}
  \\ \hline
  \multicolumn{4}{|c|}{Decompressed size} &
  \multicolumn{4}{|c}{Data... [Chunk size - 4]}
  \\ \hline
\end{tabular}

\subsubsection[MODE1 (CTF)]{MODE1 (Clickteam Fusion)}
Compressed but unencrypted.
\\
\begin{tabular}{c|c|c|c|c|c|c|c}
  0x0 & 0x1 & 0x2 & 0x3 & 0x4 & 0x5 & 0x6 & 0x7
  \\ \hline
  \multicolumn{2}{|c|}{Id} &
  0x01 & 0x00 &
  \multicolumn{4}{|c|}{Chunk size}
  \\ \hline
  \multicolumn{4}{|c|}{Decompressed size} &
  \multicolumn{4}{|c|}{Compressed size}
  \\ \hline
  \multicolumn{8}{|c}{Data... [Compressed size]}
  \\ \hline
\end{tabular}

\subsubsection{MODE2}
Uncompressed but encrypted.
\\
\begin{tabular}{c|c|c|c|c|c|c|c}
  0x0 & 0x1 & 0x2 & 0x3 & 0x4 & 0x5 & 0x6 & 0x7
  \\ \hline
  \multicolumn{2}{|c|}{Id} &
  0x02 & 0x00 &
  \multicolumn{4}{|c|}{Chunk size}
  \\ \hline
  \multicolumn{8}{|c}{Data... [Chunk size]}
  \\ \hline
\end{tabular}

\subsubsection{MODE3}
Compressed and encrypted.
\\
\begin{tabular}{c|c|c|c|c|c|c|c}
  0x0 & 0x1 & 0x2 & 0x3 & 0x4 & 0x5 & 0x6 & 0x7
  \\ \hline
  \multicolumn{2}{|c|}{Id} &
  0x03 & 0x00 &
  \multicolumn{4}{|c|}{Chunk size}
  \\ \hline
  \multicolumn{8}{|c}{Data... [Chunk size]}
  \\ \hline
\end{tabular}

\subsection{Items}
\label{subsec:items}

One of the exceptions to the chunk structure rule is item chunks (elements of
a bank chunk).
Generally speaking, items have a 32-bit identification (key) and no mode.
The mode of an item chunk is derived from the chunk bank that it is
from, rather than the item chunk itself.

\subsubsection[MMF items]{Multimedia Fusion items}

\begin{tabular}{c|c|c|c|c|c|c|c}
  0x0 & 0x1 & 0x2 & 0x3 & 0x4 & 0x5 & 0x6 & 0x7
  \\ \hline
  \multicolumn{4}{|c|}{Id} &
  \multicolumn{4}{|c|}{Decompressed size}
  \\ \hline
  \multicolumn{8}{|c}{Data... [???]}
  \\ \hline
\end{tabular}
\\
Note: It is currently assumed that the `Data' portion of the chunk must
actually be decompressed in order to figure out how large it is.
It is possible that the offset chunk in the bank
(sec~\ref{subsec:offset-chunks}) could actually store the offsets of all the
items, which could be used to derive the length of the `Data' portion.

\subsubsection[Uncompressed CTF items]{Uncompressed Clickteam Fusion items}

\begin{tabular}{c|c|c|c|c|c|c|c}
  0x0 & 0x1 & 0x2 & 0x3 & 0x4 & 0x5 & 0x6 & 0x7
  \\ \hline
  \multicolumn{4}{|c|}{Id} &
  \multicolumn{4}{|c}{Optional header...}
  \\ \hline \hline
  \multicolumn{4}{|c|}{Size} &
  \multicolumn{4}{|c}{Data... [Size]}
  \\ \hline
\end{tabular}

\subsubsection[Compressed CTF items]{Compressed Clickteam Fusion items}

\begin{tabular}{c|c|c|c|c|c|c|c}
  0x0 & 0x1 & 0x2 & 0x3 & 0x4 & 0x5 & 0x6 & 0x7
  \\ \hline
  \multicolumn{4}{|c|}{Id} &
  \multicolumn{4}{|c}{Optional header...}
  \\ \hline \hline
  \multicolumn{4}{|c|}{Decompressed Size} &
  \multicolumn{4}{|c|}{Compressed size}
  \\ \hline
  \multicolumn{8}{|c}{Data... [Compressed size]}
  \\ \hline
\end{tabular}

\subsubsection[Unknown CTF items]{Unknown Clickteam Fusion items}

\begin{tabular}{c|c|c|c|c|c|c|c}
  0x0 & 0x1 & 0x2 & 0x3 & 0x4 & 0x5 & 0x6 & 0x7
  \\ \hline
  \multicolumn{4}{|c|}{Id} &
  \multicolumn{1}{|c|}{0xFF} &
  \multicolumn{1}{|c|}{0xFF} &
  \multicolumn{1}{|c|}{0xFF} &
  \multicolumn{1}{|c|}{0xFF}
  \\ \hline
  \multicolumn{4}{|c|}{Unknown} &
  \multicolumn{1}{|c|}{0x01} &
  \multicolumn{1}{|c|}{0x00} &
  \multicolumn{1}{|c|}{0x00} &
  \multicolumn{1}{|c|}{0x00}
  \\ \hline \hline
  \multicolumn{4}{|c|}{Size} &
  \multicolumn{4}{|c}{Data... [Size + 20]}
  \\ \hline
\end{tabular}


\subsection{Banks}
\label{subsec:banks}

Unlike regular chunks, the child chunks of a bank chunk are item chunks.

% ------------------
% Common chunk types
% ------------------

\section[Common chunks]{Common chunk types}
\label{sec:common-chunks}

Many of the chunk types share a data structure, the following are the known
common structures.

\subsection[String chunk]{String chunk}
\label{subsec:string-chunk}

The `Data' of this chunk is either a single null terminated ASCII string or
a single null terminated UTF-16/USC-2 string.
See section~\ref{subsubsec:PAME} and section~\ref{subsubsec:PAMU} for more
information.

% -----------------------------
% Multimedia Fusion chunk types
% -----------------------------

\section[MMF chunks]{Multimedia Fusion chunk types}
\label{sec:mmf-chunks}

The following section describes the identification each known Multimedia
Fusion type, along with the type's uncompressed data structure.
This structure is the `Data' portion of the chunk as described in
section~\ref{subsec:chunks}.

\subsection{Game header}
\verb|OLD_HEADER = 0x2223|

\subsection{Frame items}
\verb|OLD_FRAMEITEMS = 0x2229|

\subsection{Frame items 2}
\verb|OLD_FRAMEITEMS2 = 0x223F|

\subsection{Frame}
\verb|OLD_FRAME = 0x3333|

\subsection{Frame header}
\verb|OLD_OBJHEADER = 0x3334|

\subsection{Object instance}
\verb|OLD_OBJINST = 0x3338|

\subsection{Frame events}
\verb|OLD_FRAMEEVENTS = 0x333D|

\subsection{Object properties}
\verb|OLD_OBJPROP = 0x4446|

% ----------------------------
% Clickteam Fusion chunk types
% ----------------------------

\section[CTF chunks]{Clickteam Fusion chunk types}
\label{sec:ctf-chunks}

The following section describes the identification each known Clickteam Fusion
type, along with the type's uncompressed and unencrypted data structure.
This structure is the `Data' portion of the chunk as described in
section~\ref{subsec:chunks}.

\subsection{Vitalise - 0x11XX}

\subsubsection{Vitalise preview}
\verb|VITAPREV = 0x1122|

\subsection{App - 0x22XX}

\subsubsection{Game header}
\verb|HEADER = 0x2223|

\subsubsection{Title}
\verb|TITLE = 0x2224|
\\
A string chunk (sec~\ref{subsec:string-chunk}) containing the title of the
game.

\subsubsection{Author}
\verb|AUTHOR = 0x2225|
\\
A string chunk (sec~\ref{subsec:string-chunk}) containing the author of the
game.

\subsubsection{Menu}
\verb|MENU = 0x2226|
\\
\begin{tabular}{c|c|c|c|c|c|c|c}
  0x0 & 0x1 & 0x2 & 0x3 & 0x4 & 0x5 & 0x6 & 0x7
  \\ \hline
  \multicolumn{4}{|c|}{Unknown} &
  \multicolumn{4}{|c|}{Unknown}
  \\ \hline
  \multicolumn{4}{|c|}{Unknown} &
  \multicolumn{4}{|c|}{End of string offset}
  \\ \hline
  \multicolumn{4}{|c|}{Unknown} &
  \multicolumn{4}{|c|}{Unknown}
  \\ \hline
  \multicolumn{2}{|c|}{Length1} &
  \multicolumn{6}{|c}{String1 [Length1 - 2]}
  \\ \hline \hline
  \multicolumn{2}{|c|}{LengthN} &
  \multicolumn{6}{|c}{StringN [LengthN - 2]}
  \\ \hline
  \multicolumn{8}{|c}{Unknown...}
  \\ \hline
\end{tabular}

\subsubsection{Extra path}
\verb|EXTPATH = 0x2227|

\subsubsection{Extentions}
\verb|EXTENS = 0x2228|
\\
`Deprecated'

\subsubsection{Object bank/frame items}
\verb|OBJECTBANK = 0x2229|
\\
\begin{tabular}{c|c|c|c|c|c|c|c}
  0x0 & 0x1 & 0x2 & 0x3 & 0x4 & 0x5 & 0x6 & 0x7
  \\ \hline
  \multicolumn{4}{|c|}{Item count} &
  \multicolumn{4}{|c}{Item chunks... [Item count]}
  \\ \hline
\end{tabular}


\subsubsection{Global events}
\verb|GLOBALEVENTS = 0x222A|
\\
`Deprecated'

\subsubsection{Frame handles}
\verb|FRAMEHANDLES = 0x222B|

\subsubsection{Extra data}
\verb|EXTDATA = 0x222C|

\subsubsection{Additional extentions}
\verb|ADDEXTNS = 0x222D|
\\
`Deprecated'

\subsubsection{Project path}
\verb|PROJPATH = 0x222E|
\\
A string chunk (sec~\ref{subsec:string-chunk}) containing the project path of
the game.
The project path is the path to the MFA or CCA project file that the game was
built from.

\subsubsection{Output path}
\verb|OUTPATH = 0x222F|
\\
A string chunk (sec~\ref{subsec:string-chunk}) containing the output path of
the game.
The output path is the path the author originally exported the game's
executable to.

\subsubsection{App doc}
\verb|APPDOC = 0x2230|

\subsubsection{Other extention(s)}
\verb|OTHEREXT = 0x2231|

\subsubsection{Global values}
\verb|GLOBALVALS = 0x2232|

\subsubsection{Global strings}
\verb|GLOBALSTRS = 0x2233|

\subsubsection{Extentions list}
\verb|EXTNLIST = 0x2234|

\subsubsection{Icon}
\verb|ICON = 0x2235|

\subsubsection{Demo version}
\verb|DEMOVER = 0x2236|

\subsubsection{Security number}
\verb|SECNUM = 0x2237|

\subsubsection{Binary files}
\verb|BINFILES = 0x2238|

\subsubsection{Menu images}
\verb|MENUIMAGES = 0x2239|

\subsubsection{About}
\verb|ABOUT = 0x223A|

\subsubsection{Copyright}
\verb|COPYRIGHT = 0x223B|

\subsubsection{Global value names}
\verb|GLOBALVALNAMES = 0x223C|

\subsubsection{Global string names}
\verb|GLOBALSTRNAMES = 0x223D|

\subsubsection{Movement extensions}
\verb|MOVEMNTEXTNS = 0x223E|

\subsubsection{Object bank/frame items 2}
\verb|OBJECTBANK2 = 0x223F|

\subsubsection{EXE only}
\verb|EXEONLY = 0x2240|

\subsubsection{Protection}
\verb|PROTECTION = 0x2242|

\subsubsection{Shaders}
\verb|SHADERS = 0x2243|

\subsubsection{Extended header}
\verb|EXTDHEADER = 0x2245|

\subsubsection{Spacer}
\verb|SPACER = 0x2246|

\subsubsection{Frame bank}
\verb|FRAMEBANK = 0x224D|
\\
Note: the frame item chunks inside the frame bank are not actually in the banks
data, they follow directly after the bank chunk.

\subsubsection{Chunk 224F}
\verb|CHUNK224F = 0x224F|

\subsubsection{Title 2}
\verb|TITLE2 = 0x2251|
\\
`"StringChunk" ?'

\subsubsection{Chunk 2253}
\verb|CHUNK2253 = 0x2253|
\\
`16 bytes'

\subsubsection{Chunk 2254}
\verb|CHUNK2254 = 0x2254|
\\
`Strings'

\subsubsection{Chunk 2255}
\verb|CHUNK2255 = 0x2255|
\\
`Empty'

\subsubsection{Chunk 2256}
\verb|CHUNK2256 = 0x2256|
\\
`Compressed'

\subsubsection{Chunk 2257}
\verb|CHUNK2257 = 0x2257|
\\
`4 bytes'

\subsubsection{Font Bank 2}
\verb|FONTBANK2 = 0x2258|
\\
`Fonts'

\subsubsection{Font}
\verb|FONTCHUNK = 0x2259|
\\
`Font'

\subsection{Frame chunks - 0x33XX}

\subsubsection{Frame}
\verb|FRAME = 0x3333|

\subsubsection{Frame header}
\verb|FRAMEHEADER = 0x3334|

\subsubsection{Frame name}
\verb|FRAMENAME = 0x3335|

\subsubsection{Frame password}
\verb|FRAMEPASSWORD = 0x3336|

\subsubsection{Frame palette}
\verb|FRAMEAPALETTE = 0x3337|

\subsubsection{Object instances}
\verb|OBJINST = 0x3338|

\subsubsection{Frame fade in - frame}
\verb|FRAMEFADEIF = 0x3339|

\subsubsection{Frame fade out - frame}
\verb|FRAMEFADEOF = 0x333A|

\subsubsection{Frame fade in}
\verb|FRAMEFADEI = 0x333B|

\subsubsection{Frame fade out}
\verb|FRAMEFADEO = 0x333C|

\subsubsection{Frame events}
\verb|FRAMEEVENTS = 0x333D|

\subsubsection{Frame play header}
\verb|FRAMEPLYHEAD = 0x333E|

\subsubsection{Frame additional item}
\verb|FRAMEADDITEM = 0x333F|

\subsubsection{Frame additional item instance}
\verb|FRAMEADDITEMINST = 0x3340|

\subsubsection{Frame layers}
\verb|FRAMELAYERS = 0x3341|

\subsubsection{Frame virtical size}
\verb|FRAMEVIRTSIZE = 0x3342|

\subsubsection{Frame demo file path}
\verb|DEMOFILEPATH = 0x3343|

\subsubsection{Frame random seed}
\verb|RANDOMSEED = 0x3344|

\subsubsection{Frame layer effect}
\verb|FRAMELAYEREFFECT = 0x3345|

\subsubsection{Frame BluRay options}
\verb|FRAMBLURAY = 0x3346|

\subsubsection{Frame movement timer base}
\verb|MOVETIMEBASE = 0x3347|

\subsubsection{Frame mosaic image table}
\verb|MOSAICIMGTABLE = 0x3348|

\subsubsection{Frame effects}
\verb|FRAMEEFFECTS = 0x3349|

\subsubsection{Frame iPhone options}
\verb|FRAMEIPHONEOPTS = 0x334A|

\subsubsection{Frame chunk 334C}
\verb|FRAMECHUNK334C = 0x334C|

\subsection{Object chunks - 0x44XX}

\subsubsection{Object header}
\verb|OBJHEAD = 0x4444|

\subsubsection{Object name}
\verb|OBJNAME = 0x4445|

\subsubsection{Object properties}
\verb|OBJPROP = 0x4446|

\subsubsection{Object chunk 4447}
\verb|OBJCHUNK4447 = 0x4447|

\subsubsection{Object effect}
\verb|OBJEFCT = 0x4448|

\subsection{Offect chunks - 0x55XX}
\label{subsec:offset-chunks}

\subsubsection{Image bank end}
\verb|ENDIMAGE = 0x5555|

\subsubsection{Font bank end}
\verb|ENDFONT = 0x5556|

\subsubsection{Sound bank end}
\verb|ENDSOUND = 0x5557|

\subsubsection{Music bank end}
\verb|ENDMUSIC = 0x5558|

\subsection{Bank chunks - 0x66XX}

\subsubsection{Image bank}
\verb|IMAGEBANK = 0x6666|

\subsubsection{Font bank}
\verb|FONTBANK = 0x6667|

\subsubsection{Sound bank}
\verb|SOUNDBANK = 0x6668|

\subsubsection{Music bank}
\verb|MUSICBANK = 0x6669|

\subsection{Bank items}

As described in section~\ref{subsec:items}, item chunks' identification is more
of a bank key than a type ID, hence the lack of identifications in this
section.

\subsubsection{Image item}

\subsubsection{Font item}

\subsubsection{Sound item}

\subsubsection{Music item}

\subsection{Last}
\verb|LAST = 0x7F7F|
\\
Marks the end of a data structure.

\end{document}